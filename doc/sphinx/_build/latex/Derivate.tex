%% Generated by Sphinx.
\def\sphinxdocclass{report}
\documentclass[a4paper,10pt,english]{sphinxmanual}
\ifdefined\pdfpxdimen
   \let\sphinxpxdimen\pdfpxdimen\else\newdimen\sphinxpxdimen
\fi \sphinxpxdimen=.75bp\relax
\ifdefined\pdfimageresolution
    \pdfimageresolution= \numexpr \dimexpr1in\relax/\sphinxpxdimen\relax
\fi
%% let collapsible pdf bookmarks panel have high depth per default
\PassOptionsToPackage{bookmarksdepth=5}{hyperref}

\PassOptionsToPackage{warn}{textcomp}
\usepackage[utf8]{inputenc}
\ifdefined\DeclareUnicodeCharacter
% support both utf8 and utf8x syntaxes
  \ifdefined\DeclareUnicodeCharacterAsOptional
    \def\sphinxDUC#1{\DeclareUnicodeCharacter{"#1}}
  \else
    \let\sphinxDUC\DeclareUnicodeCharacter
  \fi
  \sphinxDUC{00A0}{\nobreakspace}
  \sphinxDUC{2500}{\sphinxunichar{2500}}
  \sphinxDUC{2502}{\sphinxunichar{2502}}
  \sphinxDUC{2514}{\sphinxunichar{2514}}
  \sphinxDUC{251C}{\sphinxunichar{251C}}
  \sphinxDUC{2572}{\textbackslash}
\fi
\usepackage{cmap}
\usepackage[T1]{fontenc}
\usepackage{amsmath,amssymb,amstext}
\usepackage{babel}



\usepackage{tgtermes}
\usepackage{tgheros}
\renewcommand{\ttdefault}{txtt}



\usepackage[Bjarne]{fncychap}
\usepackage{sphinx}

\fvset{fontsize=auto}
\usepackage{geometry}


% Include hyperref last.
\usepackage{hyperref}
% Fix anchor placement for figures with captions.
\usepackage{hypcap}% it must be loaded after hyperref.
% Set up styles of URL: it should be placed after hyperref.
\urlstyle{same}


\usepackage{sphinxmessages}
\setcounter{tocdepth}{2}



\title{Derivate Documentation}
\date{Tuesday, 16 November 2021}
\release{0.1 {[}3 \sphinxhyphen{} Alpha{]}}
\author{Jan-Philipp Hoffmann}
\newcommand{\sphinxlogo}{\vbox{}}
\renewcommand{\releasename}{Release}
\makeindex
\begin{document}

\pagestyle{empty}
\sphinxmaketitle
\pagestyle{plain}
\sphinxtableofcontents
\pagestyle{normal}
\phantomsection\label{\detokenize{index::doc}}



\chapter{Introduction}
\label{\detokenize{intro:introduction}}\label{\detokenize{intro::doc}}
\noindent\sphinxincludegraphics{{logo}.png}


\section{Python Project \sphinxstyleemphasis{Derivate}}
\label{\detokenize{intro:python-project-derivate}}

\section{Introduction}
\label{\detokenize{intro:id1}}
\sphinxAtStartPar
To import the project simply type

\begin{sphinxVerbatim}[commandchars=\\\{\}]
\PYG{g+gp}{\PYGZgt{}\PYGZgt{}\PYGZgt{} }\PYG{k+kn}{import} \PYG{n+nn}{Derivate}
\end{sphinxVerbatim}

\sphinxAtStartPar
after installation.


\section{Install}
\label{\detokenize{intro:install}}
\sphinxAtStartPar
The latest stable version can always be installed or updated via pip:

\begin{sphinxVerbatim}[commandchars=\\\{\}]
\PYGZdl{} pip install Derivate
\end{sphinxVerbatim}


\section{License}
\label{\detokenize{intro:license}}
\sphinxAtStartPar
Code and documentation are available according to the license
(see LICENSE file in repository).


\chapter{Tutorial}
\label{\detokenize{tutorial:tutorial}}\label{\detokenize{tutorial::doc}}
\begin{sphinxVerbatim}[commandchars=\\\{\}]
\PYG{g+gp}{\PYGZgt{}\PYGZgt{}\PYGZgt{} }\PYG{k+kn}{import} \PYG{n+nn}{Derivate}
\end{sphinxVerbatim}

\sphinxAtStartPar
Write your tutorial here…


\chapter{API Documentation}
\label{\detokenize{doc:api-documentation}}\label{\detokenize{doc::doc}}

\section{Derivate}
\label{\detokenize{api/modules:derivate}}\label{\detokenize{api/modules::doc}}

\subsection{Derivate package}
\label{\detokenize{api/Derivate:derivate-package}}\label{\detokenize{api/Derivate::doc}}

\subsubsection{Submodules}
\label{\detokenize{api/Derivate:module-Derivate.session1_exercise3}}\label{\detokenize{api/Derivate:submodules}}\index{module@\spxentry{module}!Derivate.session1\_exercise3@\spxentry{Derivate.session1\_exercise3}}\index{Derivate.session1\_exercise3@\spxentry{Derivate.session1\_exercise3}!module@\spxentry{module}}\index{BondCashFlowLegList (class in Derivate.session1\_exercise3)@\spxentry{BondCashFlowLegList}\spxextra{class in Derivate.session1\_exercise3}}

\begin{fulllineitems}
\phantomsection\label{\detokenize{api/Derivate:Derivate.session1_exercise3.BondCashFlowLegList}}\pysigline{\sphinxbfcode{\sphinxupquote{class\DUrole{w}{  }}}\sphinxcode{\sphinxupquote{Derivate.session1\_exercise3.}}\sphinxbfcode{\sphinxupquote{BondCashFlowLegList}}}
\sphinxAtStartPar
Bases: \sphinxcode{\sphinxupquote{dcf.cashflow.CashFlowLegList}}

\sphinxAtStartPar
Simple fixed rate bond
\begin{quote}\begin{description}
\item[{Parameters}] \leavevmode\begin{itemize}
\item {} 
\sphinxAtStartPar
\sphinxstyleliteralstrong{\sphinxupquote{payment\_date\_list}} \textendash{} list of payment dates,
last date in list sets the bond maturity

\item {} 
\sphinxAtStartPar
\sphinxstyleliteralstrong{\sphinxupquote{notional\_amount}} \textendash{} bond notional amount

\item {} 
\sphinxAtStartPar
\sphinxstyleliteralstrong{\sphinxupquote{fixed\_rate}} \textendash{} fixed interest rate

\item {} 
\sphinxAtStartPar
\sphinxstyleliteralstrong{\sphinxupquote{forward\_curve}} \textendash{} InterestRateCurve

\item {} 
\sphinxAtStartPar
\sphinxstyleliteralstrong{\sphinxupquote{day\_count}} \textendash{} day count convention function
to calculate year fractions between two dates

\item {} 
\sphinxAtStartPar
\sphinxstyleliteralstrong{\sphinxupquote{start\_date}} \textendash{} issue date of the bond

\end{itemize}

\end{description}\end{quote}
\index{\_\_init\_\_() (Derivate.session1\_exercise3.BondCashFlowLegList method)@\spxentry{\_\_init\_\_()}\spxextra{Derivate.session1\_exercise3.BondCashFlowLegList method}}

\begin{fulllineitems}
\phantomsection\label{\detokenize{api/Derivate:Derivate.session1_exercise3.BondCashFlowLegList.__init__}}\pysiglinewithargsret{\sphinxbfcode{\sphinxupquote{\_\_init\_\_}}}{\emph{\DUrole{n}{payment\_date\_list}}, \emph{\DUrole{n}{notional\_amount}\DUrole{o}{=}\DUrole{default_value}{1.0}}, \emph{\DUrole{n}{fixed\_rate}\DUrole{o}{=}\DUrole{default_value}{0.0}}, \emph{\DUrole{n}{forward\_curve}\DUrole{o}{=}\DUrole{default_value}{None}}, \emph{\DUrole{n}{day\_count}\DUrole{o}{=}\DUrole{default_value}{None}}, \emph{\DUrole{n}{start\_date}\DUrole{o}{=}\DUrole{default_value}{None}}}{}
\sphinxAtStartPar
Simple fixed rate bond
\begin{quote}\begin{description}
\item[{Parameters}] \leavevmode\begin{itemize}
\item {} 
\sphinxAtStartPar
\sphinxstyleliteralstrong{\sphinxupquote{payment\_date\_list}} \textendash{} list of payment dates,
last date in list sets the bond maturity

\item {} 
\sphinxAtStartPar
\sphinxstyleliteralstrong{\sphinxupquote{notional\_amount}} \textendash{} bond notional amount

\item {} 
\sphinxAtStartPar
\sphinxstyleliteralstrong{\sphinxupquote{fixed\_rate}} \textendash{} fixed interest rate

\item {} 
\sphinxAtStartPar
\sphinxstyleliteralstrong{\sphinxupquote{forward\_curve}} \textendash{} InterestRateCurve

\item {} 
\sphinxAtStartPar
\sphinxstyleliteralstrong{\sphinxupquote{day\_count}} \textendash{} day count convention function
to calculate year fractions between two dates

\item {} 
\sphinxAtStartPar
\sphinxstyleliteralstrong{\sphinxupquote{start\_date}} \textendash{} issue date of the bond

\end{itemize}

\end{description}\end{quote}

\end{fulllineitems}


\end{fulllineitems}

\phantomsection\label{\detokenize{api/Derivate:module-Derivate.session2_exercise1}}\index{module@\spxentry{module}!Derivate.session2\_exercise1@\spxentry{Derivate.session2\_exercise1}}\index{Derivate.session2\_exercise1@\spxentry{Derivate.session2\_exercise1}!module@\spxentry{module}}\index{get\_basis\_point\_value() (in module Derivate.session2\_exercise1)@\spxentry{get\_basis\_point\_value()}\spxextra{in module Derivate.session2\_exercise1}}

\begin{fulllineitems}
\phantomsection\label{\detokenize{api/Derivate:Derivate.session2_exercise1.get_basis_point_value}}\pysiglinewithargsret{\sphinxcode{\sphinxupquote{Derivate.session2\_exercise1.}}\sphinxbfcode{\sphinxupquote{get\_basis\_point\_value}}}{\emph{\DUrole{n}{cashflow\_list}}, \emph{\DUrole{n}{discount\_curve}}, \emph{\DUrole{n}{valuation\_date}\DUrole{o}{=}\DUrole{default_value}{None}}, \emph{\DUrole{n}{delta\_curve}\DUrole{o}{=}\DUrole{default_value}{None}}, \emph{\DUrole{o}{**}\DUrole{n}{kwargs}}}{}
\sphinxAtStartPar
value of one basis point change in interest rates
\begin{quote}\begin{description}
\item[{Parameters}] \leavevmode\begin{itemize}
\item {} 
\sphinxAtStartPar
\sphinxstyleliteralstrong{\sphinxupquote{cashflow\_list}} \textendash{} cashflow product for valuation

\item {} 
\sphinxAtStartPar
\sphinxstyleliteralstrong{\sphinxupquote{discount\_curve}} \textendash{} discount curve for valuation

\item {} 
\sphinxAtStartPar
\sphinxstyleliteralstrong{\sphinxupquote{valuation\_date}} \textendash{} date of valuation

\item {} 
\sphinxAtStartPar
\sphinxstyleliteralstrong{\sphinxupquote{delta\_curve}} \textendash{} interest rate curve to change interest rates

\item {} 
\sphinxAtStartPar
\sphinxstyleliteralstrong{\sphinxupquote{kwargs}} \textendash{} additional argument for \sphinxstyleemphasis{get\_present\_value} function

\end{itemize}

\item[{Returns}] \leavevmode
\sphinxAtStartPar


\end{description}\end{quote}

\end{fulllineitems}

\phantomsection\label{\detokenize{api/Derivate:module-Derivate.session2_exercise4}}\index{module@\spxentry{module}!Derivate.session2\_exercise4@\spxentry{Derivate.session2\_exercise4}}\index{Derivate.session2\_exercise4@\spxentry{Derivate.session2\_exercise4}!module@\spxentry{module}}\index{get\_bucketed\_delta() (in module Derivate.session2\_exercise4)@\spxentry{get\_bucketed\_delta()}\spxextra{in module Derivate.session2\_exercise4}}

\begin{fulllineitems}
\phantomsection\label{\detokenize{api/Derivate:Derivate.session2_exercise4.get_bucketed_delta}}\pysiglinewithargsret{\sphinxcode{\sphinxupquote{Derivate.session2\_exercise4.}}\sphinxbfcode{\sphinxupquote{get\_bucketed\_delta}}}{\emph{\DUrole{n}{cashflow\_list}}, \emph{\DUrole{n}{discount\_curve}}, \emph{\DUrole{n}{valuation\_date}\DUrole{o}{=}\DUrole{default_value}{None}}, \emph{\DUrole{n}{delta\_curve}\DUrole{o}{=}\DUrole{default_value}{None}}, \emph{\DUrole{n}{delta\_grid}\DUrole{o}{=}\DUrole{default_value}{None}}, \emph{\DUrole{n}{shift}\DUrole{o}{=}\DUrole{default_value}{0.0001}}}{}
\sphinxAtStartPar
list of bpv delta for partly shifted interest rate curve
\begin{quote}\begin{description}
\item[{Parameters}] \leavevmode\begin{itemize}
\item {} 
\sphinxAtStartPar
\sphinxstyleliteralstrong{\sphinxupquote{cashflow\_list}} \textendash{} list of cashflows

\item {} 
\sphinxAtStartPar
\sphinxstyleliteralstrong{\sphinxupquote{discount\_curve}} \textendash{} discount factors are obtained from this curve

\item {} 
\sphinxAtStartPar
\sphinxstyleliteralstrong{\sphinxupquote{valuation\_date}} \textendash{} date to discount to

\item {} 
\sphinxAtStartPar
\sphinxstyleliteralstrong{\sphinxupquote{delta\_curve}} \textendash{} curve which will be shifted

\item {} 
\sphinxAtStartPar
\sphinxstyleliteralstrong{\sphinxupquote{delta\_grid}} \textendash{} grid dates to build partly shifts

\item {} 
\sphinxAtStartPar
\sphinxstyleliteralstrong{\sphinxupquote{shift}} \textendash{} shift size to derive bpv

\end{itemize}

\item[{Returns}] \leavevmode
\sphinxAtStartPar
\sphinxtitleref{list(float)} \sphinxhyphen{} basis point value for each \sphinxstylestrong{delta\_grid} point

\end{description}\end{quote}

\sphinxAtStartPar
Let \(v(t, r)\) be the present value of the given \sphinxstylestrong{cashflow\_list}
depending on interest rate curve \(r\)
which can be used as forward curve to estimate float rates
or as zero rate curve to derive discount factors (or both).

\sphinxAtStartPar
Then, with \sphinxstylestrong{shift\_size} \(s\) and shifting \(s_j\),

\sphinxAtStartPar
\begin{equation*}
\begin{split}\Delta_j(t) = 0.0001 \cdot \frac{v(t, r + s_j) - v(t, r)}{s}\end{split}
\end{equation*}

\sphinxAtStartPar
and the full bucketed delta vector is
\(\big(\Delta_1(t), \Delta_2(t), \dots, \Delta_{m-1}(t) \Delta_m(t)\big)\).

\sphinxAtStartPar
Overall the shifting \(s_1, \dots s_n\) is a partition of the unity,
i.e. \(\sum_{j=1}^m s_j = s\).

\sphinxAtStartPar
Each \(s_j\) for \(i=2, \dots, m-1\) is a function of the form of an triangle,
i.e. for a \sphinxstylestrong{delta\_grid} \(t_1, \dots, t_m\)
\[
s_j(t) =
\left\{
\begin{array}{cl}
    0 & \text{ for } t < t_{j-1} \\
    s \cdot \frac{t-t_{j-1}}{t_j-t_{j-1}}
        & \text{ for } t_{j-1} \leq t < t_j \\
    s \cdot \frac{t_{j+1}-t}{t_{j+1}-t_j}
        & \text{ for } t_j \leq t < t_{j+1} \\
    0 & \text{ for } t_{j+1} \leq t \\
\end{array}
\right.
\]
\sphinxAtStartPar
while
\[
s_1(t) =
\left\{
\begin{array}{cl}
    s & \text{ for } t < t_1 \\
    s \cdot \frac{t_2-t}{t_2-t_1} & \text{ for } t_1 \leq t < t_2 \\
    0 & \text{ for } t_2 \leq t \\
\end{array}
\right.
\]
\sphinxAtStartPar
and
\[
s_m(t) =
\left\{
\begin{array}{cl}
    0 & \text{ for } t < t_{m-1} \\
    s \cdot \frac{t-t_{m-1}}{t_m-t_{m-1}}
        & \text{ for } t_{m-1} \leq t < t_m \\
    s & \text{ for } t_m \leq t \\
\end{array}
\right.
\]
\end{fulllineitems}

\phantomsection\label{\detokenize{api/Derivate:module-Derivate.session3_exercise1}}\index{module@\spxentry{module}!Derivate.session3\_exercise1@\spxentry{Derivate.session3\_exercise1}}\index{Derivate.session3\_exercise1@\spxentry{Derivate.session3\_exercise1}!module@\spxentry{module}}\phantomsection\label{\detokenize{api/Derivate:module-Derivate.session3_exercise2}}\index{module@\spxentry{module}!Derivate.session3\_exercise2@\spxentry{Derivate.session3\_exercise2}}\index{Derivate.session3\_exercise2@\spxentry{Derivate.session3\_exercise2}!module@\spxentry{module}}\phantomsection\label{\detokenize{api/Derivate:module-Derivate.session3_exercise3}}\index{module@\spxentry{module}!Derivate.session3\_exercise3@\spxentry{Derivate.session3\_exercise3}}\index{Derivate.session3\_exercise3@\spxentry{Derivate.session3\_exercise3}!module@\spxentry{module}}\index{get\_bucketed\_delta() (in module Derivate.session3\_exercise3)@\spxentry{get\_bucketed\_delta()}\spxextra{in module Derivate.session3\_exercise3}}

\begin{fulllineitems}
\phantomsection\label{\detokenize{api/Derivate:Derivate.session3_exercise3.get_bucketed_delta}}\pysiglinewithargsret{\sphinxcode{\sphinxupquote{Derivate.session3\_exercise3.}}\sphinxbfcode{\sphinxupquote{get\_bucketed\_delta}}}{\emph{\DUrole{n}{cashflow\_list}}, \emph{\DUrole{n}{discount\_curve}}, \emph{\DUrole{n}{valuation\_date}\DUrole{o}{=}\DUrole{default_value}{None}}, \emph{\DUrole{n}{delta\_curve}\DUrole{o}{=}\DUrole{default_value}{None}}, \emph{\DUrole{n}{delta\_grid}\DUrole{o}{=}\DUrole{default_value}{None}}, \emph{\DUrole{n}{shift}\DUrole{o}{=}\DUrole{default_value}{0.0001}}}{}
\sphinxAtStartPar
list of bpv delta for partly shifted interest rate curve
\begin{quote}\begin{description}
\item[{Parameters}] \leavevmode\begin{itemize}
\item {} 
\sphinxAtStartPar
\sphinxstyleliteralstrong{\sphinxupquote{cashflow\_list}} \textendash{} list of cashflows

\item {} 
\sphinxAtStartPar
\sphinxstyleliteralstrong{\sphinxupquote{discount\_curve}} \textendash{} discount factors are obtained from this curve

\item {} 
\sphinxAtStartPar
\sphinxstyleliteralstrong{\sphinxupquote{valuation\_date}} \textendash{} date to discount to

\item {} 
\sphinxAtStartPar
\sphinxstyleliteralstrong{\sphinxupquote{delta\_curve}} \textendash{} curve which will be shifted

\item {} 
\sphinxAtStartPar
\sphinxstyleliteralstrong{\sphinxupquote{delta\_grid}} \textendash{} grid dates to build partly shifts

\item {} 
\sphinxAtStartPar
\sphinxstyleliteralstrong{\sphinxupquote{shift}} \textendash{} shift size to derive bpv

\end{itemize}

\item[{Returns}] \leavevmode
\sphinxAtStartPar
\sphinxtitleref{list(float)} \sphinxhyphen{} basis point value for each \sphinxstylestrong{delta\_grid} point

\end{description}\end{quote}

\sphinxAtStartPar
Let \(v(t, r)\) be the present value of the given \sphinxstylestrong{cashflow\_list}
depending on interest rate curve \(r\)
which can be used as forward curve to estimate float rates
or as zero rate curve to derive discount factors (or both).

\sphinxAtStartPar
Then, with \sphinxstylestrong{shift\_size} \(s\) and shifting \(s_j\),

\sphinxAtStartPar
\begin{equation*}
\begin{split}\Delta_j(t) = 0.0001 \cdot \frac{v(t, r + s_j) - v(t, r)}{s}\end{split}
\end{equation*}

\sphinxAtStartPar
and the full bucketed delta vector is
\(\big(\Delta_1(t), \Delta_2(t), \dots, \Delta_{m-1}(t) \Delta_m(t)\big)\).

\sphinxAtStartPar
Overall the shifting \(s_1, \dots s_n\) is a partition of the unity,
i.e. \(\sum_{j=1}^m s_j = s\).

\sphinxAtStartPar
Each \(s_j\) for \(i=2, \dots, m-1\) is a function of the form of an triangle,
i.e. for a \sphinxstylestrong{delta\_grid} \(t_1, \dots, t_m\)
\[
s_j(t) =
\left\{
\begin{array}{cl}
    0 & \text{ for } t < t_{j-1} \\
    s \cdot \frac{t-t_{j-1}}{t_j-t_{j-1}}
        & \text{ for } t_{j-1} \leq t < t_j \\
    s \cdot \frac{t_{j+1}-t}{t_{j+1}-t_j}
        & \text{ for } t_j \leq t < t_{j+1} \\
    0 & \text{ for } t_{j+1} \leq t \\
\end{array}
\right.
\]
\sphinxAtStartPar
while
\[
s_1(t) =
\left\{
\begin{array}{cl}
    s & \text{ for } t < t_1 \\
    s \cdot \frac{t_2-t}{t_2-t_1} & \text{ for } t_1 \leq t < t_2 \\
    0 & \text{ for } t_2 \leq t \\
\end{array}
\right.
\]
\sphinxAtStartPar
and
\[
s_m(t) =
\left\{
\begin{array}{cl}
    0 & \text{ for } t < t_{m-1} \\
    s \cdot \frac{t-t_{m-1}}{t_m-t_{m-1}}
        & \text{ for } t_{m-1} \leq t < t_m \\
    s & \text{ for } t_m \leq t \\
\end{array}
\right.
\]
\end{fulllineitems}



\subsubsection{Module contents}
\label{\detokenize{api/Derivate:module-Derivate}}\label{\detokenize{api/Derivate:module-contents}}\index{module@\spxentry{module}!Derivate@\spxentry{Derivate}}\index{Derivate@\spxentry{Derivate}!module@\spxentry{module}}
\sphinxAtStartPar
Derivative Finanzinstrumente (created by auxilium)
\index{BondCashFlowLegList (class in Derivate)@\spxentry{BondCashFlowLegList}\spxextra{class in Derivate}}

\begin{fulllineitems}
\phantomsection\label{\detokenize{api/Derivate:Derivate.BondCashFlowLegList}}\pysigline{\sphinxbfcode{\sphinxupquote{class\DUrole{w}{  }}}\sphinxcode{\sphinxupquote{Derivate.}}\sphinxbfcode{\sphinxupquote{BondCashFlowLegList}}}
\sphinxAtStartPar
Bases: \sphinxcode{\sphinxupquote{dcf.cashflow.CashFlowLegList}}

\sphinxAtStartPar
Simple fixed rate bond
\begin{quote}\begin{description}
\item[{Parameters}] \leavevmode\begin{itemize}
\item {} 
\sphinxAtStartPar
\sphinxstyleliteralstrong{\sphinxupquote{payment\_date\_list}} \textendash{} list of payment dates,
last date in list sets the bond maturity

\item {} 
\sphinxAtStartPar
\sphinxstyleliteralstrong{\sphinxupquote{notional\_amount}} \textendash{} bond notional amount

\item {} 
\sphinxAtStartPar
\sphinxstyleliteralstrong{\sphinxupquote{fixed\_rate}} \textendash{} fixed interest rate

\item {} 
\sphinxAtStartPar
\sphinxstyleliteralstrong{\sphinxupquote{forward\_curve}} \textendash{} InterestRateCurve

\item {} 
\sphinxAtStartPar
\sphinxstyleliteralstrong{\sphinxupquote{day\_count}} \textendash{} day count convention function
to calculate year fractions between two dates

\item {} 
\sphinxAtStartPar
\sphinxstyleliteralstrong{\sphinxupquote{start\_date}} \textendash{} issue date of the bond

\end{itemize}

\end{description}\end{quote}
\index{\_\_init\_\_() (Derivate.BondCashFlowLegList method)@\spxentry{\_\_init\_\_()}\spxextra{Derivate.BondCashFlowLegList method}}

\begin{fulllineitems}
\phantomsection\label{\detokenize{api/Derivate:Derivate.BondCashFlowLegList.__init__}}\pysiglinewithargsret{\sphinxbfcode{\sphinxupquote{\_\_init\_\_}}}{\emph{\DUrole{n}{payment\_date\_list}}, \emph{\DUrole{n}{notional\_amount}\DUrole{o}{=}\DUrole{default_value}{1.0}}, \emph{\DUrole{n}{fixed\_rate}\DUrole{o}{=}\DUrole{default_value}{0.0}}, \emph{\DUrole{n}{forward\_curve}\DUrole{o}{=}\DUrole{default_value}{None}}, \emph{\DUrole{n}{day\_count}\DUrole{o}{=}\DUrole{default_value}{None}}, \emph{\DUrole{n}{start\_date}\DUrole{o}{=}\DUrole{default_value}{None}}}{}
\sphinxAtStartPar
Simple fixed rate bond
\begin{quote}\begin{description}
\item[{Parameters}] \leavevmode\begin{itemize}
\item {} 
\sphinxAtStartPar
\sphinxstyleliteralstrong{\sphinxupquote{payment\_date\_list}} \textendash{} list of payment dates,
last date in list sets the bond maturity

\item {} 
\sphinxAtStartPar
\sphinxstyleliteralstrong{\sphinxupquote{notional\_amount}} \textendash{} bond notional amount

\item {} 
\sphinxAtStartPar
\sphinxstyleliteralstrong{\sphinxupquote{fixed\_rate}} \textendash{} fixed interest rate

\item {} 
\sphinxAtStartPar
\sphinxstyleliteralstrong{\sphinxupquote{forward\_curve}} \textendash{} InterestRateCurve

\item {} 
\sphinxAtStartPar
\sphinxstyleliteralstrong{\sphinxupquote{day\_count}} \textendash{} day count convention function
to calculate year fractions between two dates

\item {} 
\sphinxAtStartPar
\sphinxstyleliteralstrong{\sphinxupquote{start\_date}} \textendash{} issue date of the bond

\end{itemize}

\end{description}\end{quote}

\end{fulllineitems}


\end{fulllineitems}

\index{get\_basis\_point\_value() (in module Derivate)@\spxentry{get\_basis\_point\_value()}\spxextra{in module Derivate}}

\begin{fulllineitems}
\phantomsection\label{\detokenize{api/Derivate:Derivate.get_basis_point_value}}\pysiglinewithargsret{\sphinxcode{\sphinxupquote{Derivate.}}\sphinxbfcode{\sphinxupquote{get\_basis\_point\_value}}}{\emph{\DUrole{n}{cashflow\_list}}, \emph{\DUrole{n}{discount\_curve}}, \emph{\DUrole{n}{valuation\_date}\DUrole{o}{=}\DUrole{default_value}{None}}, \emph{\DUrole{n}{delta\_curve}\DUrole{o}{=}\DUrole{default_value}{None}}, \emph{\DUrole{o}{**}\DUrole{n}{kwargs}}}{}
\sphinxAtStartPar
value of one basis point change in interest rates
\begin{quote}\begin{description}
\item[{Parameters}] \leavevmode\begin{itemize}
\item {} 
\sphinxAtStartPar
\sphinxstyleliteralstrong{\sphinxupquote{cashflow\_list}} \textendash{} cashflow product for valuation

\item {} 
\sphinxAtStartPar
\sphinxstyleliteralstrong{\sphinxupquote{discount\_curve}} \textendash{} discount curve for valuation

\item {} 
\sphinxAtStartPar
\sphinxstyleliteralstrong{\sphinxupquote{valuation\_date}} \textendash{} date of valuation

\item {} 
\sphinxAtStartPar
\sphinxstyleliteralstrong{\sphinxupquote{delta\_curve}} \textendash{} interest rate curve to change interest rates

\item {} 
\sphinxAtStartPar
\sphinxstyleliteralstrong{\sphinxupquote{kwargs}} \textendash{} additional argument for \sphinxstyleemphasis{get\_present\_value} function

\end{itemize}

\item[{Returns}] \leavevmode
\sphinxAtStartPar


\end{description}\end{quote}

\end{fulllineitems}

\index{get\_bucketed\_delta() (in module Derivate)@\spxentry{get\_bucketed\_delta()}\spxextra{in module Derivate}}

\begin{fulllineitems}
\phantomsection\label{\detokenize{api/Derivate:Derivate.get_bucketed_delta}}\pysiglinewithargsret{\sphinxcode{\sphinxupquote{Derivate.}}\sphinxbfcode{\sphinxupquote{get\_bucketed\_delta}}}{\emph{\DUrole{n}{cashflow\_list}}, \emph{\DUrole{n}{discount\_curve}}, \emph{\DUrole{n}{valuation\_date}\DUrole{o}{=}\DUrole{default_value}{None}}, \emph{\DUrole{n}{delta\_curve}\DUrole{o}{=}\DUrole{default_value}{None}}, \emph{\DUrole{n}{delta\_grid}\DUrole{o}{=}\DUrole{default_value}{None}}, \emph{\DUrole{n}{shift}\DUrole{o}{=}\DUrole{default_value}{0.0001}}}{}
\sphinxAtStartPar
list of bpv delta for partly shifted interest rate curve
\begin{quote}\begin{description}
\item[{Parameters}] \leavevmode\begin{itemize}
\item {} 
\sphinxAtStartPar
\sphinxstyleliteralstrong{\sphinxupquote{cashflow\_list}} \textendash{} list of cashflows

\item {} 
\sphinxAtStartPar
\sphinxstyleliteralstrong{\sphinxupquote{discount\_curve}} \textendash{} discount factors are obtained from this curve

\item {} 
\sphinxAtStartPar
\sphinxstyleliteralstrong{\sphinxupquote{valuation\_date}} \textendash{} date to discount to

\item {} 
\sphinxAtStartPar
\sphinxstyleliteralstrong{\sphinxupquote{delta\_curve}} \textendash{} curve which will be shifted

\item {} 
\sphinxAtStartPar
\sphinxstyleliteralstrong{\sphinxupquote{delta\_grid}} \textendash{} grid dates to build partly shifts

\item {} 
\sphinxAtStartPar
\sphinxstyleliteralstrong{\sphinxupquote{shift}} \textendash{} shift size to derive bpv

\end{itemize}

\item[{Returns}] \leavevmode
\sphinxAtStartPar
\sphinxtitleref{list(float)} \sphinxhyphen{} basis point value for each \sphinxstylestrong{delta\_grid} point

\end{description}\end{quote}

\sphinxAtStartPar
Let \(v(t, r)\) be the present value of the given \sphinxstylestrong{cashflow\_list}
depending on interest rate curve \(r\)
which can be used as forward curve to estimate float rates
or as zero rate curve to derive discount factors (or both).

\sphinxAtStartPar
Then, with \sphinxstylestrong{shift\_size} \(s\) and shifting \(s_j\),

\sphinxAtStartPar
\begin{equation*}
\begin{split}\Delta_j(t) = 0.0001 \cdot \frac{v(t, r + s_j) - v(t, r)}{s}\end{split}
\end{equation*}

\sphinxAtStartPar
and the full bucketed delta vector is
\(\big(\Delta_1(t), \Delta_2(t), \dots, \Delta_{m-1}(t) \Delta_m(t)\big)\).

\sphinxAtStartPar
Overall the shifting \(s_1, \dots s_n\) is a partition of the unity,
i.e. \(\sum_{j=1}^m s_j = s\).

\sphinxAtStartPar
Each \(s_j\) for \(i=2, \dots, m-1\) is a function of the form of an triangle,
i.e. for a \sphinxstylestrong{delta\_grid} \(t_1, \dots, t_m\)
\[
s_j(t) =
\left\{
\begin{array}{cl}
    0 & \text{ for } t < t_{j-1} \\
    s \cdot \frac{t-t_{j-1}}{t_j-t_{j-1}}
        & \text{ for } t_{j-1} \leq t < t_j \\
    s \cdot \frac{t_{j+1}-t}{t_{j+1}-t_j}
        & \text{ for } t_j \leq t < t_{j+1} \\
    0 & \text{ for } t_{j+1} \leq t \\
\end{array}
\right.
\]
\sphinxAtStartPar
while
\[
s_1(t) =
\left\{
\begin{array}{cl}
    s & \text{ for } t < t_1 \\
    s \cdot \frac{t_2-t}{t_2-t_1} & \text{ for } t_1 \leq t < t_2 \\
    0 & \text{ for } t_2 \leq t \\
\end{array}
\right.
\]
\sphinxAtStartPar
and
\[
s_m(t) =
\left\{
\begin{array}{cl}
    0 & \text{ for } t < t_{m-1} \\
    s \cdot \frac{t-t_{m-1}}{t_m-t_{m-1}}
        & \text{ for } t_{m-1} \leq t < t_m \\
    s & \text{ for } t_m \leq t \\
\end{array}
\right.
\]
\end{fulllineitems}

\index{Line (class in Derivate)@\spxentry{Line}\spxextra{class in Derivate}}

\begin{fulllineitems}
\phantomsection\label{\detokenize{api/Derivate:Derivate.Line}}\pysigline{\sphinxbfcode{\sphinxupquote{class\DUrole{w}{  }}}\sphinxcode{\sphinxupquote{Derivate.}}\sphinxbfcode{\sphinxupquote{Line}}}
\sphinxAtStartPar
Bases: \sphinxcode{\sphinxupquote{object}}

\sphinxAtStartPar
This a example class (by auxilium)

\sphinxAtStartPar
The {\hyperref[\detokenize{api/Derivate:Derivate.Line}]{\sphinxcrossref{\sphinxcode{\sphinxupquote{Derivate.Line}}}}} objects implements a straight line,
i.e. a function \(y = f(x)\) with

\sphinxAtStartPar
\$\$  f(x) = a + b \textbackslash{}cdot x  \$\$

\sphinxAtStartPar
where \(a\) and \(b\) are numbers.

\begin{sphinxVerbatim}[commandchars=\\\{\}]
\PYG{g+gp}{\PYGZgt{}\PYGZgt{}\PYGZgt{} }\PYG{k+kn}{from} \PYG{n+nn}{Derivate} \PYG{k+kn}{import} \PYG{n}{Line}
\PYG{g+gp}{\PYGZgt{}\PYGZgt{}\PYGZgt{} }\PYG{n}{a}\PYG{p}{,} \PYG{n}{b} \PYG{o}{=} \PYG{l+m+mi}{1}\PYG{p}{,} \PYG{l+m+mi}{2}
\PYG{g+gp}{\PYGZgt{}\PYGZgt{}\PYGZgt{} }\PYG{n}{line} \PYG{o}{=} \PYG{n}{Line}\PYG{p}{(}\PYG{n}{a}\PYG{p}{,} \PYG{n}{b}\PYG{p}{)}
\PYG{g+gp}{\PYGZgt{}\PYGZgt{}\PYGZgt{} }\PYG{n}{line}\PYG{o}{.}\PYG{n}{y}\PYG{p}{(}\PYG{n}{x}\PYG{o}{=}\PYG{l+m+mi}{3}\PYG{p}{)}
\PYG{g+go}{7}
\PYG{g+gp}{\PYGZgt{}\PYGZgt{}\PYGZgt{} }\PYG{n}{line}\PYG{p}{(}\PYG{l+m+mi}{3}\PYG{p}{)}  \PYG{c+c1}{\PYGZsh{} Line objects are callable}
\PYG{g+go}{7}
\PYG{g+gp}{\PYGZgt{}\PYGZgt{}\PYGZgt{} }\PYG{n}{line}\PYG{o}{.}\PYG{n}{a}
\PYG{g+go}{1}
\PYG{g+gp}{\PYGZgt{}\PYGZgt{}\PYGZgt{} }\PYG{n}{line}\PYG{o}{.}\PYG{n}{b}
\PYG{g+go}{2}
\end{sphinxVerbatim}
\begin{quote}\begin{description}
\item[{Parameters}] \leavevmode\begin{itemize}
\item {} 
\sphinxAtStartPar
\sphinxstyleliteralstrong{\sphinxupquote{a}} \textendash{} a value

\item {} 
\sphinxAtStartPar
\sphinxstyleliteralstrong{\sphinxupquote{b}} \textendash{} another value

\end{itemize}

\end{description}\end{quote}
\index{\_\_init\_\_() (Derivate.Line method)@\spxentry{\_\_init\_\_()}\spxextra{Derivate.Line method}}

\begin{fulllineitems}
\phantomsection\label{\detokenize{api/Derivate:Derivate.Line.__init__}}\pysiglinewithargsret{\sphinxbfcode{\sphinxupquote{\_\_init\_\_}}}{\emph{\DUrole{n}{a}\DUrole{o}{=}\DUrole{default_value}{0}}, \emph{\DUrole{n}{b}\DUrole{o}{=}\DUrole{default_value}{1}}}{}~\begin{quote}\begin{description}
\item[{Parameters}] \leavevmode\begin{itemize}
\item {} 
\sphinxAtStartPar
\sphinxstyleliteralstrong{\sphinxupquote{a}} \textendash{} a value

\item {} 
\sphinxAtStartPar
\sphinxstyleliteralstrong{\sphinxupquote{b}} \textendash{} another value

\end{itemize}

\end{description}\end{quote}

\end{fulllineitems}

\index{a (Derivate.Line property)@\spxentry{a}\spxextra{Derivate.Line property}}

\begin{fulllineitems}
\phantomsection\label{\detokenize{api/Derivate:Derivate.Line.a}}\pysigline{\sphinxbfcode{\sphinxupquote{property\DUrole{w}{  }}}\sphinxbfcode{\sphinxupquote{a}}}
\sphinxAtStartPar
a value

\end{fulllineitems}

\index{b (Derivate.Line property)@\spxentry{b}\spxextra{Derivate.Line property}}

\begin{fulllineitems}
\phantomsection\label{\detokenize{api/Derivate:Derivate.Line.b}}\pysigline{\sphinxbfcode{\sphinxupquote{property\DUrole{w}{  }}}\sphinxbfcode{\sphinxupquote{b}}}
\sphinxAtStartPar
b value

\end{fulllineitems}

\index{y() (Derivate.Line method)@\spxentry{y()}\spxextra{Derivate.Line method}}

\begin{fulllineitems}
\phantomsection\label{\detokenize{api/Derivate:Derivate.Line.y}}\pysiglinewithargsret{\sphinxbfcode{\sphinxupquote{y}}}{\emph{\DUrole{n}{x}\DUrole{o}{=}\DUrole{default_value}{1}}}{}
\sphinxAtStartPar
gives y value depending on x value argument
\begin{quote}\begin{description}
\item[{Parameters}] \leavevmode
\sphinxAtStartPar
\sphinxstyleliteralstrong{\sphinxupquote{x}} \textendash{} x value

\item[{Returns}] \leavevmode
\sphinxAtStartPar
\(a + b * x\)

\end{description}\end{quote}

\end{fulllineitems}


\end{fulllineitems}



\chapter{Releases}
\label{\detokenize{releases:releases}}\label{\detokenize{releases::doc}}
\sphinxAtStartPar
These changes are listed in decreasing version number order.


\section{Release 0.1}
\label{\detokenize{releases:release-0-1}}
\sphinxAtStartPar
Release date was Tuesday, 16 November 2021


\chapter{Indices and tables}
\label{\detokenize{index:indices-and-tables}}\begin{itemize}
\item {} 
\sphinxAtStartPar
\DUrole{xref,std,std-ref}{genindex}

\item {} 
\sphinxAtStartPar
\DUrole{xref,std,std-ref}{modindex}

\item {} 
\sphinxAtStartPar
\DUrole{xref,std,std-ref}{search}

\end{itemize}


\renewcommand{\indexname}{Python Module Index}
\begin{sphinxtheindex}
\let\bigletter\sphinxstyleindexlettergroup
\bigletter{d}
\item\relax\sphinxstyleindexentry{Derivate}\sphinxstyleindexpageref{api/Derivate:\detokenize{module-Derivate}}
\item\relax\sphinxstyleindexentry{Derivate.session1\_exercise3}\sphinxstyleindexpageref{api/Derivate:\detokenize{module-Derivate.session1_exercise3}}
\item\relax\sphinxstyleindexentry{Derivate.session2\_exercise1}\sphinxstyleindexpageref{api/Derivate:\detokenize{module-Derivate.session2_exercise1}}
\item\relax\sphinxstyleindexentry{Derivate.session2\_exercise4}\sphinxstyleindexpageref{api/Derivate:\detokenize{module-Derivate.session2_exercise4}}
\item\relax\sphinxstyleindexentry{Derivate.session3\_exercise1}\sphinxstyleindexpageref{api/Derivate:\detokenize{module-Derivate.session3_exercise1}}
\item\relax\sphinxstyleindexentry{Derivate.session3\_exercise2}\sphinxstyleindexpageref{api/Derivate:\detokenize{module-Derivate.session3_exercise2}}
\item\relax\sphinxstyleindexentry{Derivate.session3\_exercise3}\sphinxstyleindexpageref{api/Derivate:\detokenize{module-Derivate.session3_exercise3}}
\end{sphinxtheindex}

\renewcommand{\indexname}{Index}
\printindex
\end{document}